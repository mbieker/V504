\documentclass[11pt,ngerman,a4paper]{article}
%Gummi|061|=)
\usepackage{amsmath}
\usepackage{a4wide}
\usepackage{url}
\usepackage{amsthm}
\usepackage{amsbsy}
\usepackage{amssymb}
\usepackage[utf8]{inputenc}
\usepackage{rotating} 
\usepackage{here}
\usepackage{graphicx}
\usepackage{paralist}
\usepackage{selinput}
\usepackage[separate-uncertainty=true]{siunitx}
\usepackage{booktabs}
\sisetup{}
\SelectInputMappings{%
adieresis={ä},
germandbls={ß},
}
\title{\textbf{Versuch V504: Thermische Elektronenemission}}
\author{Martin Bieker\\
		Julian Surmann\\
		\\
		Durchgef\"{u}hrt am 17.06.2014\\
		TU Dortmund}
\date{}
\usepackage{graphicx}
\begin{document}
\renewcommand\tablename{Tabelle}
\renewcommand\figurename{Abbildung}
\maketitle
\thispagestyle{empty}
\newpage
\clearpage
\setcounter{page}{1}


\section{Einleitung}
In diesem Versuch wird die sogenannte Thermische Elektronenemission untersucht, dabei handelt es sich um die Auslösung von Elektronen aus einer Metalloberfläche mit Hilfe von thermischer Energie. Die Temperaturabhängigkeit dieses Vorganges ist von besonderem Interesse.
\section{Theorie}
\section{Auswertung}
\section{Diskussion}

\section{Quellen}
\begin{enumerate}[{[}1{]}]
\item Entnommen der Praktikumsanleitung \textit{} der TU Dortmund. \\
Download am 01.06.14 unter:\\
 \url{http://129.217.224.2/HOMEPAGE/PHYSIKER/BACHELOR/AP/SKRIPT/V703.pdf}
\end{enumerate}

\section{Anhang}
\begin{itemize}
\item Tabellen
\item Auszug aus dem Messheft
\end{itemize}
\end{document}
